\documentclass[a4paper,10pt,landscape]{article}
\usepackage{multicol}
\usepackage{calc}
\usepackage{ifthen}
\usepackage[landscape]{geometry}
\usepackage{amsmath,amsthm,amsfonts,amssymb}
\usepackage{color,graphicx,overpic}
\usepackage{hyperref}
\usepackage[defaultsans]{droidsans}
\usepackage[defaultmono]{droidmono}
\renewcommand*\familydefault{\sfdefault} %% Only if the base font of the document is to be typewriter style
\usepackage[T1]{fontenc}
\usepackage[utf8]{inputenc}
\usepackage{tabularx}

\pdfinfo{
  /Title (vim-cheatsheet.pdf)
  /Creator (LaTeX)
  /Author (Jens Dede)
  /Subject (vi)
  /Keywords (vi, vim, gvim)}

% This sets page margins to .5 inch if using letter paper, and to 1cm
% if using A4 paper. (This probably isn't strictly necessary.)
% If using another size paper, use default 1cm margins.
\ifthenelse{\lengthtest { \paperwidth = 11in}}
    { \geometry{top=.5in,left=.5in,right=.5in,bottom=.5in} }
    {\ifthenelse{ \lengthtest{ \paperwidth = 297mm}}
        {\geometry{top=1cm,left=1cm,right=1cm,bottom=1cm} }
        {\geometry{top=1cm,left=1cm,right=1cm,bottom=1cm} }
    }

% Turn off header and footer
\pagestyle{empty}

% Redefine section commands to use less space
\makeatletter
\renewcommand{\section}{\@startsection{section}{1}{0mm}%
                                {-1ex plus -.5ex minus -.2ex}%
                                {0.5ex plus .2ex}%x
                                {\normalfont\large\bfseries}}
\renewcommand{\subsection}{\@startsection{subsection}{2}{0mm}%
                                {-1explus -.5ex minus -.2ex}%
                                {0.5ex plus .2ex}%
                                {\normalfont\normalsize\bfseries}}
\renewcommand{\subsubsection}{\@startsection{subsubsection}{3}{0mm}%
                                {-1ex plus -.5ex minus -.2ex}%
                                {1ex plus .2ex}%
                                {\normalfont\small\bfseries}}
\makeatother

% Define BibTeX command
\def\BibTeX{{\rm B\kern-.05em{\sc i\kern-.025em b}\kern-.08em
    T\kern-.1667em\lower.7ex\hbox{E}\kern-.125emX}}

% Don't print section numbers
\setcounter{secnumdepth}{0}


\setlength{\parindent}{0pt}
\setlength{\parskip}{0pt plus 0.5ex}

%My Environments
\newtheorem{example}[section]{Example}
% -----------------------------------------------------------------------

\begin{document}
\raggedright
\footnotesize
\begin{multicols}{3}


% multicol parameters
% These lengths are set only within the two main columns
%\setlength{\columnseprule}{0.25pt}
\setlength{\premulticols}{1pt}
\setlength{\postmulticols}{1pt}
\setlength{\multicolsep}{1pt}
\setlength{\columnsep}{2pt}

\begin{center}
     \Large{\underline{Basic vim commands}} \\
     {\tiny
         (Based on the .vimrc from \url{http://github.com/jdede/tools/blob/master/vimrc/vimrc})
     }
\end{center}

\section{Find and replace}
\tiny{
    \begin{tabularx}{\columnwidth}{|l|X|}
        \hline
        \texttt{:/<expr>}&Forward search for <expr>\\
        \texttt{:?<expr>}&Backward search for <expr>\\
        \texttt{n}&Go to next appearance of <expr>\\
        \hline
        \texttt{:\%s/<find>/<replace>/}&Replace <find> by <replace>\\
        \texttt{:\%s/<find>/<replace>/g}&Replace every <find> by <replace>\\
        \hline
    \end{tabularx}
}

\section{Tabs and spaces}
\tiny{With my \texttt{.vimrc} tabs are replaces by spaces. To switch the
behavior on the fly, the following commands are quite useful.}
\tiny{
    \begin{tabularx}{\columnwidth}{|l|X|}
        \hline
        \texttt{:set noexpandtab}&Do not replace a tab by spaces\\
        \texttt{:set expandtab}&Replace tab by spaces\\
        \hline
        \texttt{:set tabstop}&Print number of spaces to replace a tab\\
        \texttt{:set tabstop=<x>}&Every tab is replaced by <x> spaces\\
        \hline
        \texttt{:retab}&Format current document with the current settings
        (regarding to tabs and spaces)\\
        \hline
    \end{tabularx}
}

\section{CR / LF formatting}
\tiny{Documents are opened in unix format by default. Windows Line endings are
only detected, if all lines ends are the same. So files are opened in unix
format as default and can be changed on demand.}
\tiny{
    \begin{tabularx}{\columnwidth}{|l|X|}
        \hline
        \texttt{:e ++ff=dos}&Reload file with dos line endings\\
        \texttt{:e ++ff=unix}&Reload file with unix line endings\\
        \hline
        \texttt{:w ++ff=dos}&Save with dos line endings\\
        \hline
    \end{tabularx}
}

\section{Split view}
\tiny{Split screen and work with the split view.}
\tiny{
    \begin{tabularx}{\columnwidth}{|l|X|}
        \hline
        \texttt{:split}&Open current used file in horizontal split view\\
        \texttt{:split <filename>}&Open new file in horizontal split view\\
        \texttt{:vsplit}&Open file in vertical split mode\\
        \hline
        \texttt{:only}&Close all split views but the active one\\
        \texttt{ctrl-w <arrowkey>}&Move through the different views\\
        \texttt{10 ctrl-w+}&Increase window size by 10 lines\\
        \hline
    \end{tabularx}
}

\section{Commenting source code}
\tiny{How to block comment / uncomment source code.}
\tiny{
    \begin{tabularx}{\columnwidth}{|l|X|}
        \hline
        \texttt{ctrl-v <mark block> I \% ESC}&Comment block with \%\\
        \texttt{ctrl-v <mark block> x}&Remove vertical aligned comment symbols\\
        \hline
    \end{tabularx}
}

\section{Marker}
\tiny{How to mark a position in the text and move back later.}
\tiny{
    \begin{tabularx}{\columnwidth}{|l|X|}
        \hline
        \texttt{m[a-Z]}&Set marker with id a-Z. Lower case: in file, upper
        case: unique in all open files\\
        \texttt{'[a-Z]}&Jump to marked position\\
        \hline
    \end{tabularx}
}

\section{Execute external commands}
\tiny{How to start an external shell or execute a command without leaving vim.}
\tiny{
    \begin{tabularx}{\columnwidth}{|l|X|}
        \hline
        \texttt{:sh}&Open a shell. Type exit to return to vim\\
        \hline
        \texttt{:! <command>}&Execute <command> in shell. Return to vim after
        command has finished\\
        \texttt{:! <command> \%}&Execute command with current edited file as
        parameter\\
        \hline
    \end{tabularx}
}

\end{multicols}
\end{document}
